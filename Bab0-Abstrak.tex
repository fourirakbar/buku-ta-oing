\begin{abstrak}
	\indent Saat ini penggunaan kontainer docker dalam dunia teknologi sangat banyak dilakukan. Kontainer docker merupakan operating-system-level virtualization untuk menjalankan beberapa sistem linux yang terisolasi (kontainer) pada sebuah host. Kontainer berfungsi untuk mengisolasi aplikasi atau servis dan dependensinya. Untuk setiap servis atau aplikasi yang terisolasi dibutuhkan satu kontainer pada server host yang ada dan setiap kontainer akan menggunakan sumber daya yang ada pada server host selama kontainer tersebut menyala.\\
    	\indent Dalam kasus ini, setiap user yang mengakses atau menggunakan jaringan ITS merupakan satu servis yang nantinya akan dibuatkan satu kontainer pada server host. Hal ini dapat mempermudah manajemen dari masing-masing user, contohnya manajemen bandwith, hak akses, waktu, dan lain sebagainya. Jika user telah selesai mengakses atau menggunakan jaringan ITS, maka kontainer pada user tersebut akan di-destroy, sehingga hal ini dapat meringankan beban server.\\

\noindent \textbf{Kata-Kunci}:  Internet Access Management, Kontainer, Docker
\end{abstrak}

\cleardoublepage
\begin{abstract}
	\indent Nowadays, docker containers have been widely used in the word of technology. THe docker containers is an operating system level virtualization to run some isolated linux systems (containers) on a host. Containers are used to isolate applications or services and its depedencies. For every service or app that isolated it takes one container on the existing host server and each container will use the existing resources on the host server as long as the container is on.
	
	In this case, any user accessing or using ITS network is one service that a container will be created on the host server. This can smplify management of each user, for exaple bandiwth management, access rights, time, and many more. If the user has finished accessing or using the the network ITS, then the container on the user will be destroyed, so this can reduce the server load. \\

\noindent \textbf{Keywords}:  \textit{Internet Access Management}, Container, Docker.
\end{abstract}