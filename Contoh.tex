% \section{Struktur Dokumen \LaTeX{}}
% Dokumen \LaTeX{} terdiri dari struktur yang dibuat berdasarkan struktur dokumen sehari-hari. Sebagai penulis dokumen, Anda wajib menggunakan struktur ini sehingga \LaTeX{} dapat melakukan hal lain yang membantu Anda dalam mengorganisir dokumen seperti misalnya pembuatan Daftar Isi. Berikut adalah struktur dokumen yang ada di \LaTeX{} diurutkan berdasarkan hirarkinya.

% \begin{ltabulary}{|L|L|} % L = Rata kiri untuk setiap kolom, | = garis batas vertikal.

% % Kepala tabel, berulang di setiap halaman
% \caption{Struktur hirarki dokumen \LaTeX{}} \label{tabelStrukturDokumen} \\
% \hline
% \textbf{Nama} & \textbf{Peruntukkan} \\ \hline

% \endhead
% \endfoot
% \endlastfoot

% % Isi Tabel
% \textbf{\textbackslash{}part\{Judul Bagian\}} & \texttt{book} \\ \hline
% \textbf{\textbackslash{}chapter\{Judul Bab\}} & \texttt{book} dan \texttt{report} \\ \hline
% \textbf{\textbackslash{}section\{Judul Subbab\}} & semua kecuali \texttt{letter} \\ \hline
% \textbf{\textbackslash{}subsection\{Judul Subsubbab\}} & semua kecuali \texttt{letter} \\ \hline
% \textbf{\textbackslash{}subsubsection\{Judul Subsubsubbab\}} & semua kecuali \texttt{letter} \\ \hline
% \textbf{\textbackslash{}paragraph\{Judul Paragraf\}} & semua\\ \hline

% \end{ltabulary}
